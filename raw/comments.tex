\section{Comments}

\subsection{Service Strategy~\cite{ITILServiceStrategy}}
Service portfolio management is directly informed
if a new strategy has been devised or an existing strategy is being changed.
The intention is not only to merely state the changes,
but also to trigger further analyses about potential impact.
We assume feedback from design, build and transition teams
might consist of
merely informing about the current status,
warning about potential problems,
complaining about problems,
and recommending certain actions.


\subsection{Service Design~\cite{ITILServiceDesign}}
Service review meetings are held on a regular basis with customers~\cite[p. 116]{ITILServiceDesign},
and service level agreements are reviewed, renegotiated, agreed upon
based on past issues and anticipated changes.
Negotiations may consist of many speech acts,
\eg based on Schoop et al~\cite{SchoopJertilaList2003}:
Request, Offer, Counter-Offer, Accept, Reject, Question, Clarification.
%
Reviews in availability management use input from service level management and IT service continuity management. Reports may contain statements about availability of services, forecasts, and recommendations. Review-based speech acts triggering availability management should, regardless of the applied speech act verb, either be assertive or directive.
%
Reviews in capacity management and IT service continuity management are not further elaborated. Since they depend on similar data and share the same wording as availability management, we adopted categorization of those triggers.
%
In IT service continuity management,
attendance of change advisory board meetings should primarily yield
notifications about changes.
%
In Supplier management,
reviews are based on a variety of input data, \eg business and customer feedback, incident reviews, service performance. Their nature also can be assertive or directive.
Requirements for new, renewed or terminated contracts
are based on assertive and commissive statements of stakeholders
as well as directive interactions.
For commitments on requirements, we used the verb \emph{confirm}.

\subsection{Service Transition~\cite{ITILServiceTransition}}
Change management is triggered by decisions, requests from users, and requests from continual service improvement.
%
Service asset and configuration management can be triggered by updates from change management.
For example, notifications, authorizations, and rejections.
Updates from release and deployment management
seem to be restricted to notifications~\cite[p. 147]{ITILServiceTransition}.
%
Knowledge management can be triggered by
business relationship management (BRM) storing the minutes of a customer meeting.
While the minutes may contain many speech acts,
BRM informs about the results.
Hence, Searle's $F(P)$ framework could be applied and the verb \emph{inform} will suffice.


\subsection{Service Operation~\cite{ITILServiceOperation}}
In incident management, potential failures raised by suppliers or technical staff
may simply be notifications (assertive) or warnings with intention of producing action (directive).
Users primarily notify current problems, demand/request solving the problem or explaining procedures, or complain about the service.
%
In problem management,
testing and suppliers might notify about unresolved issues
or warn about potential faults and known deficiencies
prior to going ahead with a release.
%
All triggers in access management are requests.

\subsection{Continual Service Improvement~\cite{ITILCSI}}
CSI triggers are only very shortly introduced.
Complaints are derived from``poor performance'' and ``spiralling costs''
due to the many interfaces to the CSI process.
