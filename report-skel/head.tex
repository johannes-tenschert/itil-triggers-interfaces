\documentclass[a4paper, 12pt, oneside]{scrbook}
\usepackage[dvipsnames]{xcolor}
\usepackage{soul}
\usepackage{longtable}
\usepackage{amsmath}

%%%%%%%%%%%%%%%%%%%%%%%%%%%%%%%%%%%%%%%%%%%%%%%%%%%%%%%%%%%%%%%%%%%%%%%%%%%%%%%
% Fancy Highlight                                                             %
%%%%%%%%%%%%%%%%%%%%%%%%%%%%%%%%%%%%%%%%%%%%%%%%%%%%%%%%%%%%%%%%%%%%%%%%%%%%%%%

\usepackage{tikz}
\usetikzlibrary{calc}
\usetikzlibrary{decorations.pathmorphing}

\makeatletter

\newcommand{\defhighlighter}[3][]{%
  \tikzset{every highlighter/.style={color=#2, fill opacity=#3, #1}}%
}

\defhighlighter{yellow}{.5}

\newcommand{\highlight@DoHighlight}{
  \fill [ decoration = {random steps, amplitude=1pt, segment length=15pt}
        , outer sep = -15pt, inner sep = 0pt, decorate
        , every highlighter, this highlighter ]
        ($(begin highlight)+(0,8pt)$) rectangle ($(end highlight)+(0,-3pt)$) ;
}

\newcommand{\highlight@BeginHighlight}{
  \coordinate (begin highlight) at (0,0) ;
}

\newcommand{\highlight@EndHighlight}{
  \coordinate (end highlight) at (0,0) ;
}

\newdimen\highlight@previous
\newdimen\highlight@current

\DeclareRobustCommand*\highlight[1][]{%
  \tikzset{this highlighter/.style={#1}}%
  \SOUL@setup
  %
  \def\SOUL@preamble{%
    \begin{tikzpicture}[overlay, remember picture]
      \highlight@BeginHighlight
      \highlight@EndHighlight
    \end{tikzpicture}%
  }%
  %
  \def\SOUL@postamble{%
    \begin{tikzpicture}[overlay, remember picture]
      \highlight@EndHighlight
      \highlight@DoHighlight
    \end{tikzpicture}%
  }%
  %
  \def\SOUL@everyhyphen{%
    \discretionary{%
      \SOUL@setkern\SOUL@hyphkern
      \SOUL@sethyphenchar
      \tikz[overlay, remember picture] \highlight@EndHighlight ;%
    }{%
    }{%
      \SOUL@setkern\SOUL@charkern
    }%
  }%
  %
  \def\SOUL@everyexhyphen##1{%
    \SOUL@setkern\SOUL@hyphkern
    \hbox{##1}%
    \discretionary{%
      \tikz[overlay, remember picture] \highlight@EndHighlight ;%
    }{%
    }{%
      \SOUL@setkern\SOUL@charkern
    }%
  }%
  %
  \def\SOUL@everysyllable{%
    \begin{tikzpicture}[overlay, remember picture]
      \path let \p0 = (begin highlight), \p1 = (0,0) in \pgfextra
        \global\highlight@previous=\y0
        \global\highlight@current =\y1
      \endpgfextra (0,0) ;
      \ifdim\highlight@current < \highlight@previous
        \highlight@DoHighlight
        \highlight@BeginHighlight
      \fi
    \end{tikzpicture}%
    \the\SOUL@syllable
    \tikz[overlay, remember picture] \highlight@EndHighlight ;%
  }%
  \SOUL@
}
\makeatother

%%%%%%%%%%%%%%%%%%%%%%%%%%%%%%%%%%%%%%%%%%%%%%%%%%%%%%%%%%%%%%%%%%%%%%%%%%%%%%%
% END Fancy Highlight                                                         %
%%%%%%%%%%%%%%%%%%%%%%%%%%%%%%%%%%%%%%%%%%%%%%%%%%%%%%%%%%%%%%%%%%%%%%%%%%%%%%%

% Necessary
%\newcommand{\hlc}[3]{\noindent\sethlcolor{#1}\textcolor{#2}{\hl{#3}}} % boring & fast
\newcommand{\hlc}[3]{\noindent\highlight[#1]{#3}} % fancy & slow

\title{Classification of Triggers in ITIL \\ (Generated Report)}
\author{Johannes Tenschert, and Richard Lenz}
\publishers{ % Short intro on title page...
	\normalfont\normalsize%
	\parbox{0.8\linewidth}{%
		This report consists of two parts.
		First,
		we identify six categories of triggers
		and list classifications for each trigger.
		Afterwards,
		for triggers that were immediately identified as speech acts,
		we outline potential illocutionary forces (intentions).
	}
}


\begin{document}

\maketitle

\chapter{Classification of Triggers}

We identified six categories of triggers:

\newcommand{\itil}[3]{\item \hlc{#2}{#3}{#1:} \\}
\newcommand{\eg}{e.\,g.\ }
\newcommand{\ie}{i.\,e.\ }
\begin{enumerate}
	\itil{Speech acts immediately identifiable}{red}{white}
	\emph{Interactions} that are some sort of speech act, \eg requesting, complaining, reporting, and triggers after review meetings. Triggers after meetings are raised through the interactions and decisions in a meeting, so even though their exact categorization was not elaborated, the actual trigger \emph{is} a speech act.
	\itil{Decision made by other department}{cyan}{white}
	A decision made by some other department has to be communicated to start a process. Even if it is only written in a document (artifact), commanding its execution definitely is a speech act. Hence, we view decisions as implicit speech acts as soon as they are communicated to put into practice.
	\itil{Artifact}{lime}{black}
	The creation of artifacts, \eg utilization rates, a scheduled meeting, or a published customer satisfaction survey, may or may not be triggered by speech acts. For reasons of clarity, artifacts do not count as speech acts.
	\itil{System event}{magenta}{white}
	Special case of artifact in which a computer system raises an exception or yields certain events and alerts. Even though some of those messages could be viewed as speech acts, for reasons of clarity they are not.
	\itil{Periodic trigger}{yellow}{black}
	For example, monthly/quarterly reporting or budget planning cycles. Periodic triggers do not count as speech acts.
	\itil{Observation}{brown}{white}
	Some insight a member of an organizational unit gained about changes in business needs, opportunities, etc. This knowledge does not always have to be transferred and interactions are not always necessary to trigger a process. For reasons of clarity, observations do not count as speech acts.
\end{enumerate}


Some parts of a trigger are just highlighted without any intended classification. Those are marked \hlc{black}{white}{like this block$_{\mbox{\tiny none}}$} and not included in any statistics.

\newpage

\vspace{10pt}
\noindent\textbf{Statistics for five core ITIL publications:}

